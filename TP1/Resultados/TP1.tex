\documentclass[]{article}

\usepackage[utf8]{inputenc}
\usepackage{float}

%opening
\title{Trabajo Práctico 1}
\author{Nicolás Muschitiello \\
	Roberto Velez\\
	Jairo Jiménez
	}

\begin{document}

\maketitle

\section{Introducción}

\section{Preprocesamiento}

\section{Software utilizado}
Para la elaboración del trabajo, se usó el software estadístico R \cite{RCran}.

\section{Medidas}
Para el análisis de las reglas interesantes, se decidió utilizar medidas adicionales a las vistas en clase. Las medidas que fueron elegidas, tienen la propiedad de ser \textit{null-invariantes}, es decir que no se ven  elegidas fueron tomadas del libro  \cite{Hall:2009:WDM:1656274.1656278}\\


Medida de Kulczynsky:

\begin{center}
	$Kulc(A,B)= \frac{1}{2}\left(P(A|B) P(B|A)\right)$	
\end{center}

Razón de desbalanceo:

\begin{center}
	$IR = \frac{|sup(A)- sup(B)|}{sup(A)+ sup(B)-sup(A \cup B)}$	
\end{center}
	


\section{Resultados técnicos esperados}

\section{Resultados no técnicos esperados}

\section{Bonus}

\bibliographystyle{apalike}
\bibliography{Bibliografia}


\end{document}
